\begin{hcarentry}{dimensional: Statically checked physical dimensions}
\label{dimensional}
\report{Bj\"orn Buckwalter}%05/09
\status{active, mostly stable}
\makeheader

Dimensional is a library providing data types for performing
arithmetics with physical quantities and units. Information about
the physical dimensions of the quantities/units is embedded in their
types, and the validity of operations is verified by the type checker
at compile time. The boxing and unboxing of numerical values as
quantities is done by multiplication and division with units. The
library is designed to, as far as is practical, enforce/encourage
best practices of unit usage.

The core of dimensional is stable with additional units being added
on an as-needed basis. In addition to the \textsc{si} system of
units, dimensional has experimental support for user-defined dimensions
and a proof-of-concept implementation of the \textsc{cgs} system
of units. I am also experimenting with forward automatic differentiation
and rudimentary linear algebra.

The current release is compatible with \textsc{ghc} 6.6.x and
above and can be downloaded from Hackage or the project web site.
The primary documentation is the literate Haskell source code, but
the wiki on the project web site has a few usage examples to help
with getting started.

\FurtherReading
 \url{https://github.com/bjornbm/dimensional}
\end{hcarentry}
