\begin{hcarentry}{dimensional}
\label{dimensional}
\report{Bjorn Buckwalter}
\status{active, mostly stable}
\entry{new entry}
\makeheader

Dimensional is a library providing data types for performing
arithmetic with physical quantities and units. Information about
the physical dimensions of the quantities/units is embedded in their
types and the validity of operations is verified by the type checker
at compile time. The boxing and unboxing of numerical values as
quantities is done by multiplication and division with units. The
library is designed to, as far as is practical, enforce/encourage
best practices of unit usage.

Following a reorganization of the module hierarchy the core of
dimensional is now mostly stable while additional units are being
added on an as-needed basis. In addition to the \textsc{si} system
of units dimensional has experimental support for user-defined
dimensions and a proof-of-concept implementation of the \textsc{cgs}
system of units.

The most recent release is compatible with \textsc{ghc} 6.6.x and
above and can be downloaded from hackage or the project web site.
The primary documentation is the literate haskell source code but
the wiki on the project web site has a few usage examples to help
with getting started.

\FurtherReading
 \url{http://dimensional.googlecode.com}
\end{hcarentry}

